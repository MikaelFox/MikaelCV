\cvsection{Andre aktiviteter}\\

\cveventone{2011}{Dog Power Adventures}{Prince George,\\ British Columbia}{Arbejdede som dyrepasser/hundetræner, kørte og trænede slædehunde, kørte med turister både med slædehunde og firhjulede motorcykler m.m. En pragtfuld oplevelse midt i Rocky Mountains fantastiske natur.}

\cveventone{2012}{Game Academy. }{Vallekilde Højskole}{Arbejdede med spiludvikling, grafisk design og programmering. Producerede flere PC spil i forløbet. Havde yderligere valgfag om EU – opbygning, struktur og organisation, samt musikforståelse med fokus på teori, og matematikkens og psykologiens betydning for, hvorfor og om vi kan lide musikken eller ikke.}

\cvsection{Lidt om mig selv}\\
Jeg er opvokset på en gård i nærheden af Viborg, hvor jeg og min bror helt fra vi var børn, har indgået i de forskellige arbejdsopgaver, der altid er på et landbrug. Begge mine forældre havde fuldtidsjobs (ingeniør og forstander), og i en del af fritiden arbejdede hele familien som et team, for at få de nødvendige opgaver løst. 

Det har givet mig mulighed for at træffe selvstændige beslutninger og tage ansvaret for dem, fra en meget tidlig alder. 

Jeg har altid nydt at færdes i naturen, både som spejder, på telture, skiferier m.v. 
Senere har min naturinteresse mere og mere rettet sig mod oplevelser i de arktiske egne. 
I påsken 2016 var jeg på Svalbard, for igen at nyde den fantastiske natur og køre med slædehunde og snescooter.  

Samtidig har jeg fra 6 års alderen, haft en stor interesse i computerteknologi og matematik. 

Jeg har deltaget i diverse computerstævner, været IT support for venner og familie, selv arrangeret LAN parties m.m. 

I 2. og 3.g deltog jeg i projektet ”Science talenter”- et projekt for særligt udvalgte studerende fra de naturvidenskabelige linjer som endte ud i en camp på Videnscentret i Sorø, finansieret af Mærskfonden. 

I gymnasietiden var jeg også med i andre naturvidenskabelige camps, Game Development Camp, Software Development Camp og Matematik Camp, igennem organisationen UNF. 
